\documentclass[a4paper,12pt]{book}
\onecolumn
\usepackage{amssymb,amsmath,amsfonts}
\usepackage[french]{babel}
%\usepackage[latin1]{inputenc}
\usepackage[T1]{fontenc}
%\usepackage{kpfonts}
\usepackage{anttor}
\usepackage{calligra}
%\usepackage{mathaccents}
\usepackage[left=2cm,right=2cm,top=2cm,bottom=2cm]{geometry}
%\usepackage[easylists]
\usepackage{color}
\usepackage{layout}
\usepackage{graphicx}
\usepackage{floatflt}
\usepackage{tablists}
%\documentclass[14.5pt,a4paper]{book}%definie les caractéristique etla classe du document
\usepackage[utf8]{inputenc}%pour inserrer les caractèes francophone
%\usepackage[french]{babel}% les accent et d'autr caractères
%\usepackage[T1]{fontenc} %pour les fonts ou police
\usepackage{parallel}
\usepackage{cwpuzzle}%pour les mots croisés
%\usepackage{amsmath}%formules et symbole maths
%\usepackage[]{graphicx} % pour inclure des graphiquue dans les docs
\usepackage{color}%colorer les encadrer
\usepackage{bclogo}%pour inséré un logo dans une boite
\usepackage{xkeyval}
\usepackage{watermark}%pour mettre en filigrane
\usepackage{xcolor}%pour avoir les differentes couleur 
\usepackage{caption}%pour placer une image dans la marge d'un texte
%\usepackage[pdftex]{graphics}
%\usepackage{N.D.T}
%\usepackage[colorgrid]{eso-pic}%permet d'obtenir du papier carrereler
%\usepackage{lmodern}%latin moderne pour des caractères latine
\usepackage{wrapfig}%inséré une image dans un texte
\usepackage{photo}%inséré des des photos
\usepackage{afterpage}%permet de stoker les commandes qui seront utiliser sur la page suivante
\usepackage{framed}%pour la personnalisation des boites et encadrer des textes
\usepackage{fancybox}%pour crée des boites
\usepackage{ulem}%pour souligner les partie du texte
\usepackage{multicol}% pour separer une feuille en plusieurs colonnes
\usepackage{fancyhdr}%pour inserer les entêtes et les pieds de pages
\pagestyle{fancy}%pour choisir un style de page personaliser
\linespread{1.6}%permet d'espacer les lignes des différents paragraphe
%\voffset = 0
%\hoffset = 0
\fancyhead{} % vide l’en-tête
%\fancyhead[RO,LE]{\bfseries Ann\'{e}e Scolaire 2014-2015}%ecrit se trouvant sur l'entête
\fancyfoot{} % vide le pied~de~page
%\fancyfoot[LE,RO]{\thepage}
%\fancyfoot[LO,CE]{Prof :Armand Ndinga }
%\fancyfoot[CO,RE]{Et : Brice Manga}
%\renewcommand{\headrulewidth}{0.4pt}%épaisseur de la ligne qui sépare l'entête du corps du doc
%\renewcommand{\footrulewidth}{0.4pt}%épaisseur de la ligne qui sépare le pied du corps du doc
\begin{document}
	\newcommand{\R}{\mathbb{R}}
	\newcommand{\F}{\mathcal{F}}
	\newcommand{\PO}{\mathbb{P}}
	\newcommand{\E}{\mathbb{E}}
	\newcommand{\SQ}{\mathcal{S}}
	\newcommand{\N}{\mathcal{N}}
	\newcommand{\dia}{\diamondsuit}
	\newcommand{\B}{\mathcal{B}}
	\newcommand{\M}{\mathcal{M}}
	\newcommand{\G}{\mathcal{G}}
	\newcommand{\LK}{\mathcal{L}}
	\newcommand{\C}{\mathcal{C}}
	%\newcommand{\i}{\mathrm{i}}
	%\newcommand{\e}{\mathrm{e}}
	\newtheorem{theo}{Th\'{e}or\`{e}me}[chapter]
	\newtheorem{lem}{Lemme}[chapter]
	\newtheorem{coro}{Corollaire}[chapter]
	\newtheorem{propo}{Proposition}[chapter]
	\newtheorem{defi}{D\'{e}finition}[chapter]
	\newtheorem{rem}{Remarque}[chapter]
	\newtheorem{exa}{Exemple}[chapter]
	\newtheorem{proof}{Preuve}[chapter]
	\newtheorem{exo}{Exercice}[chapter]
	%\tableofcontents
	%\title{Sur le Langrangien de TONELLI}
	%\author{Abdoulaye SARR}
	%\maketitle
	%\chapter{calcul dans }
	%\begin{flushleft}
	%Cours ANNE MARIE JAVOUHEY\\
	%Second Cycle
	%\end{flushleft}
	%\pagebreak
	%\thispagestyle{plain}
	%\fancypagestyle{plain}
	%\pagenumbering{Alph}
	\pagestyle{fancy}
	\lhead{\bfseries  École Catholique\\
		Saint Pierre\\}
	\chead{\bfseries Céllule de \\ Mathématiques\\ $1^{\text{ère}}$ $S_1$}
	\rhead{\bfseries Ann\'{e}e Scolaire \\ 2017-2018\\  }
	\lfoot{ }
	\cfoot{}
	\rfoot{\thepage}
	\renewcommand{\headrulewidth}{0.5pt}
	\renewcommand{\footrulewidth}{0.5pt}
	\vspace*{0.5cm}
	\begin{minipage}{2cm}
		\includegraphics[scale=0.2]{LOGO.png} 
	\end{minipage}
	\begin{minipage}{13cm}
		\begin{center}
			\textbf{COMPOSITION DU SECOND SEMESTRE DE MATHÉMATIQUES}\\
			\textbf{Durée : 4 heures}	
		\end{center}
	\end{minipage}
	
\textbf{\underline{Exercice 1:} 5 points}\\
Soit $a \in \mathbb{R}\setminus \{0;1 \}$, on considère les suites $(U_n)$ et $(V_n)$ définies par:
$\begin{cases}
U_0=0 \textbf{;} U_1=1 \\
U_{n+1}=(1-a)U_n + aU_{n-1}	
\end{cases}$\\
et $V_n=U_{n+1}-U_n$.
\begin{enumerate}
	\item Calculer  $U_2$, $U_3$ et $U_4$ en fonction de a.
	\item Calculer en fonction de a les quatre premiers termes de $(V_n)$.
	\item
	\begin{enumerate}
		\item Démontrer que $(V_n )$ est une suite géométrique de raison $q=-a$.
		\item Exprimer $V_n$ en fonction n.
		\item Déterminer les valeurs de a pour lesquelles $(V_n)$ est une suite convergente.
		\item Étudier suivant les valeurs de a le sens de variation de la suite $(V_n)$.
	\end{enumerate}
	\item On pose $a=-\displaystyle{\frac{1}{2}}$.
	\begin{enumerate}
		\item Calculer $S_n=V_0+V_1+...+V_{n-1}$ en fonction de n.
		\item Exprimer $U_n$ en fonction de $S_n$.\\
		Calculer $\displaystyle{\lim_{n \to + \infty}U_n}$.
	\end{enumerate}
\end{enumerate}

\textbf{\underline{EXERCICE 2 :} 5 points }\\
On considère un triangle ABC direct. On appelle I, J et K les milieux respectifs [BC], [CA] et [AB]. Soit N l'image de C par la rotation de centre J et d'angle $\displaystyle{\frac{\pi}{2}}$ et P l'image de A par la rotation de centre K et d'angle $\displaystyle{\frac{\pi}{2}}$.
\begin{enumerate}
	\item Montrer que KP = IJ .\textbf{ (0.75 pt)}
	\item On appelle r la rotation qui transforme K en J et P en I . Quelle est une mesure de l’angle de cette rotation ?\textbf{ (0.75 pt)}
	\item Démontrer que les triangles IJN et IKP sont isométriques. En déduire l’image de I par r et que le triangle PIN est isocèle et rectangle en I .\textbf{( 1.5 pt)}
	\item On appelle $r_1$ la rotation de centre N et d’angle $\displaystyle{\frac{\pi}{2}}$ 	et $r_2$ la rotation de centre P et d’angle $\displaystyle{\frac{\pi}{2}}$.\\
	Déterminer l'image de B par $r_1\circ r_2$.\textbf{(1 pt)} \\
	Donner la nature et les éléments caractéristiques de la transformation  $r_1\circ r_2$.\textbf{(1 pt)}
\end{enumerate}
\textbf{\underline{PROBLÈME} 10 points}\\
Le plan est muni d'un  repère orthonormé $(O,\overrightarrow{i},\overrightarrow{j})$. Soit f la fonction définie sur par $\displaystyle{f(x)=\frac{3x^2+ax+b}{x^2+1}}$.
\begin{enumerate}
	\item Déterminer les réels a et b pour que la courbe représentative de la fonction f soit tangente à la droite $(D):4x+3$ au point $I(0;3)$.
	\item Dans la suite on suppose $a=4$ et $b=3$.
	\begin{enumerate}
		\item Déterminer les réels $\alpha$ et $\beta$ tels que : $\displaystyle{f(x)=\alpha+\frac{\beta x}{x^2+1}}$.
		\item Calculer les limites en $-\infty$ et en $+\infty$ de f. Interpréter graphiquement ces résultats.
	\end{enumerate}
	\item Dresser le tableau de variation de f.
	\item 
	\begin{enumerate}
		\item Montrer que le point $I(0,3)$ est un centre de symétrie de la courbe de f.
		\item Étudier la position relative de $\mathcal{C}_f$ par rapport à la droite (D).
	\end{enumerate}
	\item Construire $\mathcal{C}_f$.
	\item Soit g la fonction définie sur $\mathbb{R}$ par :
	$\begin{cases}
	g(x)=f(x) \text{ si } x<0 \\
	g(x)=3 +\sqrt{x^2+x} \text{ si } x\geq 0	 
	\end{cases}$
	\begin{enumerate}
		\item Montrer que g est continue en 0.
		\item Étudier la dérivabilité de g en 0. Interpréter graphiquement ce résultat.
		\item Montrer  que $\mathcal{C}_g$ admet une asymptote oblique en $+\infty$. Préciser son équation.	
	\end{enumerate}
	\item Dresser le tableau de variation de g.
	\item  Tracer la courbe $\mathcal{C}_g$ dans le repère précédent.
\end{enumerate}
\end{document}